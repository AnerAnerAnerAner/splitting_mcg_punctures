\section{Proof of Theorem \ref{thm:main}}
\ref{sec:sseq}
We will prove Theorem \ref{thm:main} by induction on $m$.
The case $m=0$ is trivial.

For all $m\geq 0$ we let $E(m)$ be the Leray-Serre spectral sequence associated with
the bundle \eqref{eq:BirmanbundleD}: its second page has the form
\[
 E(m)^2_{k,q}=H_k(B\Diff(\S;\partial\S\cup\D);H_q(\cms))=H_k(\gg;H_q(\cms)).
\]
From Theorem \ref{thm:Hbms*as*ggrep} we know that this spectral sequence is concentrated
on the rows $q=0,\dots, m$.

We want to prove the vanishing of all differentials appearing in the pages
$E(m)^r$ with $r\geq 2$; the $r$-th differential takes the form
\[
 \partial_r\colon E(m)^r_{k,q}\to E(m)^r_{k-r,q+r-1}.
\]
In particular any differential $\partial_r$ exiting from the row $q=m$ is trivial,
because it lands in a higher, hence trivial row.

Fix now $q=m-l<m$, in particular $l\geq 1$; by Theorem \ref{thm:Hbms*as*ggrep}, and
in particular by Lemma \ref{lem:oplussplitting}, we have a splitting of $H_k(\gg;H_{m-l}(\cms))$ as
\begin{equation}
\label{eq:Hksplitting}
 \bigoplus_{p=l}^m H_k\pa{\gg;\mu_*\pa{H_{p-l}(C_p(\D))\otimes H_{m-p}(C_{m-p}(\S'))}}.
\end{equation} 
We fix now $l\leq p\leq m$ and show the vanishing of all differentials $\partial_r$ exiting from the
summand with label $p$ in the previous equation.

Consider the map $\mu^{\cF}$ from Definition \ref{defn:universalSbundle} as a map of
bundles over the space $B\Diff(\S;\partial\S\cup\D)$:
\[
 \mu^{\cF}\colon C_p(\D)\times C_{m-p}(\cF_{\S'})\to C_m(\cF_{\S,\D}).
\]
Note that the first bundle $C_p(\D)\times C_{m-p}(\cF_{\S'})\to B\Diff(\S;\partial\S\cup\D)$
is the product of the space $C_p(\D)$ with the bundle $C_{m-p}(\cF_{\S'})\to B\Diff(\S;\partial\S\cup\D)$;
therefore the spectral sequence associated with $C_p(\D)\times C_{m-p}(\cF_{\S'})$ is isomorphic,
from the second page on, to the tensor product
of $H_*(C_p(\D))$ and the spectral sequence associated with the bundle $C_{m-p}(\cF_{\S'})$; the
latter spectral sequence is isomorphic, in our notation, to the spectral sequence $E(m-p)$.
In particular $\mu^{\cF}$ induces a map of spectral
sequences
\[
\mu^{\cF}_*\colon H_*(C_p(\D))\otimes E(m-p)\to E(m);
\]
that in the second page, on the $(m-l)$-th row and $k$-th column, restricts to
the inclusion of one of the direct summands in equation \eqref{eq:Hksplitting}:
\[
H_k\pa{\gg;\mu_*(H_{p-l}(C_p(\D))\otimes H_{m-p}(C_{m-p}(\S')))}\subset H_k\pa{\gg;H_{m-l}(\cms)}.
\]

In particular if we prove the vanishing of all differentials $\partial_r$
% exiting from the summand
% $H_k\pa{\gg;H_{p-l}(C_p(\D))\otimes H_{m-p}(C_{m-p}(\S'))}$
in the first spectral sequence, then also all differentials
$\partial_r$ exiting from this direct summand in the
second spectral sequence $E(m)$ must vanish.

% We observe that there is an isomorphism
% \[
%  H_k\pa{\gg;H_{p-l}(C_p(\D))\otimes H_{m-p}(C_{m-p}(\S'))}\simeq H_{p-l}(C_p(\D))\otimes H_k\pa{\gg; H_{m-p}(C_{m-p}(\S'))};
% \]
The differentials in the spectral sequence $H_*(C_p(\D))\otimes E(m-p)$
are obtained by tensoring the identity of $H_*(C_p(\D))$ with the differentials
of the spectral sequence $E(m-p)$; as $p\geq l\geq 1$ we know by inductive hypothesis
that the latter vanish. Theorem \ref{thm:main} is proved.

One can generalise Theorem \ref{thm:main} to orientable or non-orientable
surfaces with non-empty boundary, following the generalisation
of Theorem \ref{thm:Hbms*as*ggrep} discussed at the end of section \ref{sec:Actiongg}.