\section{Introduction}
Let $\sg$ be a smooth orientable surface of genus $g$ with one boundary component $\partial\sg$, and let $\sgm$ be $\sg$ with
a choice of $m$ distinct points in the interior, called \emph{punctures}.

Let $\gg$ be the mapping class group of $\sg$, i.e. the group of isotopy classes of diffeomorphisms of $\sg$:
diffeomorphisms are required to fix $\partial\sg$ pointwise; similarly let $\ggm$ be the mapping class group of $\sgm$, i.e.
the group of isotopy classes of diffeomorphisms of $\sgm$ that fix $\partial\sgm$ pointwise and \emph{permute} the $m$ punctures.

Forgetting punctures yields a surjective map $\ggm\to\gg$ with kernel $\bms$,
the $m-$th \emph{braid group} of the surface $\sg$: we get the Birman exact sequence (see \cite{Birman:mcgbr}):
\begin{equation}
\label{eq:Birman}
1\to\bms\to\ggm\to\gg\to 1.
\end{equation}
% This short exact sequence corresponds to 
% \begin{equation}
%  \cms\to\mgm\to\mg.
% \end{equation}
% Here $\cms$ is the \emph{unordered} configuration space of $m$ points in the interior of $\Sigma$, i.e.
% \begin{equation}
%  \cms=\fms/\S_m=\set{(p_1,\dots,p_m)\in\pa{\mathring\sg}^m|p_i\neq p_j \;\forall i\neq j}/\S_m,
% \end{equation}
% the quotient of the \emph{ordered} configuration space of $m$ points in the interior of $\Sigma$ by the natural
% action of the symmetric group $\S_m$ that permutes the (ordered list of) points in the configuration; $\mg$ and $\mgm$
% are the moduli spaces of Riemann structures on $\sg$ and $\sgm$ respectively.

The associated Leray-Serre spectral sequence in $\Z_2-$homology has a second page $E^2_{p,q}=H_p(\gg;H_q(\bms;\Z_2))$,
and converges to $H_{p+q}(\ggm;\Z_2)$.

Our main theorem is that the spectral sequence has already collapsed in its second page.
\begin{thm}
\label{thm:main}
For each $l\geq 0$ there is an isomorphism
\begin{equation}
\label{eq:main}
 H_l(\ggm;\Z_2)\cong \bigoplus_{p+q=l} H_p(\gg;H_q(\bms;\Z_2)).
\end{equation}
\end{thm}
Thus the computation of $H_*(\ggm;\Z_2)$ reduces to the computation of the homology of $\gg$ with
twisted coefficients in the representation $H_*(\bms;\Z_2)$: as we will see the $\ggm-$representation
$H_*(\bms;\Z_2)$ splits as a direct sum of symmetric powers of $H_1(\sg)$.
In particular this representation is symplectic: this had already benn stated in \cite{LM}.

The group $\gg$ is significantly smaller than $\ggm$, so the
computation of the right-hand side of equation \ref{eq:main}
should be more tractable.
WE DO THIS COMPUTATION FOR $g=1$.

From now on $\Z_2$-coefficients will always be understood if no coefficients are specified.
