\section{Comparison with the work of B\"{o}digheimer and Tillmann}
\label{sec:comparison}
In this section we will compare Theorems \ref{thm:main} and \ref{thm:Hbms*as*ggrep}
with the results in \cite{BoT}. In particular we consider the following two theorems, that
we formulate in an equivalent, but different way as in \cite{BoT}, fitting in our
framework. In the entire section homology is taken with coefficients in $\Z_2$ unless explicitly
stated otherwise.
\begin{thm}[Corollary 1.2 in \cite{BoT}]
\label{thm:BoTone}
Let $\F$ be a field, and let $m\geq 0$, $g\geq 0$. Then the following graded vector spaces are isomorphic in degrees $*\leq \frac 23 g-\frac 23$
\[
 H_*\pa{\ggm;\F}\cong H_*\pa{\gg;\F}\otimes_{\F} H_*\pa{\mathfrak{S}_m;\F[x_1,\dots,x_m]}.
\]
Here each variable $x_1,\dots,x_m$ has degree 2 and the symmetric group $\mathfrak{S}_m$ acts on the polynomial
ring $\F[x_1,\dots,x_m]$ by permuting the variables.
\end{thm}
We point out that the range of degrees $*\leq \frac 23 g-\frac 23$ is the stable range for the homology $H_*(\gg;\F)$
of the mapping class group, see \cite{Harer} for the original stability theorem and \cite{Boldsen, ORW:resolutions_homstab} for the improved
stability range.
\begin{thm}[Theorem 1.3 in \cite{BoT}]
\label{thm:BoTtwo}
For all $m\geq 0$ the map $\mu^{\cF}\colon C_1(\D)\times C_m(\cF_{\S'})\to C_{m+1}(\cF_{\S,\D})$ from Definition \ref{defn:muF}
induces a split-injective map in homology
\[
 \mu^{\cF}_*\colon H_*\pa{C_m(\cF_{\S'})}\cong H_0\pa{C_1(\D)}\otimes_{\F} H_*\pa{C_m(\cF_{\S'})}\hookrightarrow H_*\pa{C_{m+1}(\cF_{\S,\D})}.
\]
In other words, the inclusion of groups $\ggm\hookrightarrow\gg^{m+1}$ given by adding a puncture near the boundary induces a split
injective map $H_*(\ggm;\F)\hookrightarrow H_*(\gg^{m+1};\F)$.
\end{thm}

We first reformulate Theorems \ref{thm:main} and \ref{thm:Hbms*as*ggrep} in a convenient way.
\begin{defn}
 \label{defn:Cbullet}
 We introduce some abbreviations for the following disjoint unions:
 \[
 C_{\bullet}(\D)=\coprod_{m\geq 0}C_m(\D);
 \]
 \[
 C_{\bullet}(\S)=\coprod_{m\geq 0}C_m(\S); 
 \]
 \[
 C_{\bullet}(\cF_{\S,\D})=\coprod_{m\geq 0}C_m(\cF_{\S,\D}).
 \]
\end{defn}
Then $C_{\bullet}(\D)$ is a (homotopy associative) topological monoid, with associated
Pontryagin ring $H_*\pa{C_{\bullet}(\D)}\cong\Z_2[Q^j\epsilon|j\geq 0]$ (see Equation \ref{eq:Cohen}).
This is a bigraded ring, where the two gradings are the homological degree and the weight.

The maps $\mu$ from Definition \ref{defn:muF} make $C_{\bullet}(\S)$ into a (homotopy associative)
module over the monoid $C_{\bullet}(\D)$; correspondingly $H_*\pa{C_{\bullet}(\S)}$ is a bigraded
module over $H_*\pa{C_{\bullet}(\D)}$. The structure of this module is described by the following
reformulation of Theorem \ref{thm:Hbms*as*ggrep}, see in particular the proof of Lemma \ref{lem:oplussplitting}.

\begin{thm}
 \label{thm:firstreformulation}
 There is an isomorphism of $\gg$-representations in (bigraded) modules over the ring $H_*\pa{C_{\bullet}(\D)}$
 \[
  H_*\pa{C_{\bullet}(\S)}\cong   H_*\pa{C_{\bullet}(\D)}\otimes\Sym_{\bullet}(\H).
 \]
\end{thm}

Similarly, Theorem \ref{thm:main} can be reformulated as follows, considering at the same time all values of $m\geq 0$.
\begin{thm}
 \label{thm:mainreformulation}
 There is an isomorphism of (bigraded) modules over $H_*\pa{C_{\bullet}(\D)}$
 \[
 H_*\pa{C_{\bullet}(\cF_{\S,\D})}\cong   H_*\pa{C_{\bullet}(\D)}\otimes H_*\pa{\gg;\Sym_{\bullet}(\H)}.
 \] 
\end{thm}
The proof follows from the arguments used in Section \ref{sec:sseq}: we have actually shown that the direct
sum of the spectral sequences $\bigoplus_{m\geq 0} E(m)$ is itself a free module over $H_*\pa{C_{\bullet}(\D)}$
on the second page, with the same description as above.

By virtue of homological stability in $g$ for the sequences of groups $\pa{\gg}_{g\geq 0}$ and 
$\pa{\ggm}_{g\geq 0}$, Theorem \ref{thm:BoTone} can be equivalently rephrased as an isomorphism of graded vector spaces
\[
  H_*\pa{\Gamma_{\infty,1}^m;\F}\cong H_*\pa{\Gamma_{\infty,1};\F}\otimes_{\F} H_*\pa{\mathfrak{S}_m;\F[x_1,\dots,x_m]}.
\]
Here $\Gamma_{\infty,1}=\colim_{g\to\infty}\gg$ and $\Gamma_{\infty,1}^m=\colim_{g\to\infty}\ggm$.

Let $\Gamma_{\infty,1}^{\bullet}$ denote the disjoint union $\coprod_{m\geq 0}\Gamma_{\infty,1}^m$, which we
consider as a groupoid. Then Theorem \ref{thm:BoTone} is equivalent to the isomorphism of bigraded vector spaces
\[
  H_*\pa{\Gamma_{\infty,1}^{\bullet};\F}\cong H_*\pa{\Gamma_{\infty,1};\F}\otimes_{\F} \pa{\bigoplus_{m\geq 0} H_*\pa{\mathfrak{S}_m;\F[x_1,\dots,x_m]}}.
\]
We consider $\bigoplus_{m\geq 0} H_*\pa{\mathfrak{S}_m;\F[x_1,\dots,x_m]}$ as a free module over the Pontryagin ring
$H_*(\fS_{\bullet};\F)$, where $\fS_{\bullet}=\coprod_{m\geq 0}\fS_m$ and the Pontryagin product is induced by
the natural maps of groups $\fS_m\times\fS_{m'}\to\fS_{m+m'}$. This module structure is induced by the natural maps
\[
H_*\pa{\fS_m;\F[x_1,\dots,x_m]}\otimes_{\F} H_*\pa{\fS_{m'};\F}\to  H_*\pa{\fS_m\times\fS_{m'};\F[x_1,\dots,x_m]\otimes_{\F}\F} \to
\]
\[
\to H_*\pa{\fS_{m+m'};\F[x_1,\dots,x_{m+m'}]}. 
\]
From now on we assume $\F=\Z_2$.
Let $\beta_m=\beta_m(\D)$ denote the braid group on $m$ strands of the disc, and let $\beta_{\bullet}=\coprod_{m\geq 0}\beta_m$.
Then the natural projections $\beta_m\to\fS_m$ induce an \emph{inclusion} of Pontryagin rings (see \cite{CLM})
\[
 H_*(C_{\bullet}(\D))\cong H_*(\beta_{\bullet})\hookrightarrow H_*(\fS_{\bullet}).
\]
Hence $\bigoplus_{m\geq 0} H_*\pa{\mathfrak{S}_m;\Z_2[x_1,\dots,x_m]}$ becomes a free module over $H_*(C_{\bullet}(\D))$.
Theorem \ref{thm:mainreformulation} gives then the following isomorphism, which turns out to be an isomorphism of free
$H_*(C_{\bullet}(\D))$-modules.
\begin{equation}
 \label{eq:doubleiso}
\begin{split}
  H_*\pa{\Gamma_{\infty,1}^{\bullet}} & \cong H_*(C_{\bullet}(\D))\otimes H_*\pa{\Gamma_{\infty,1};\Sym_{\bullet}(\H_{\infty})}\\
  & \cong H_*\pa{\Gamma_{\infty,1}}\otimes \pa{\bigoplus_{m\geq 0} H_*\pa{\mathfrak{S}_m;\Z_2[x_1,\dots,x_m]}}.
  \end{split}
\end{equation}
Here $\H_{\infty}=\colim_{g\to\infty} H_1(\Sigma_{g,1})$ is the first homology of the surface of infinite genus with one
boundary component.

Recall that $H_*\pa{\Gamma_{\infty,1}}$ is also a Pontryagin ring: there are natural maps of groups
$\Gamma_{g,1}\times\Gamma_{g',1}\to\Gamma_{g+g',1}$ inducing a Pontryagin product on the homology 
$H_*\pa{\Gamma_{\infty,1}}=\colim_{g\to\infty} H_*\pa{\gg}$.

Equation \ref{eq:doubleiso} gives
an isomorphism of modules over the ring
$H_*\pa{\Gamma_{\infty,1}}\otimes H_*(C_{\bullet}(\D))$. The action of the ring $H_*\pa{\Gamma_{\infty,1}}$
on $H_*\pa{\Gamma_{\infty,1}^{\bullet}}$ is induced on the colimit by 
the maps of groups $\gg\times\Gamma_{g',1}^m\to\Gamma_{g+g',1}^m$.
The corresponding action of $H_*\pa{\Gamma_{\infty,1}}$ on $H_*(C_{\bullet}(\D))\otimes H_*\pa{\Gamma_{\infty,1};\Sym_{\bullet}(\H)}$
comes from the action on $ H_*\pa{\Gamma_{\infty,1};\Sym_{\bullet}(\H)}$ which is induced on the colimit by the natural maps
\[
 H_*(\gg)\otimes H_*(\Gamma_{g',1};\Sym_m(H_1(\Sigma_{g',1})))\to H_*(\Gamma_{g+g',1};\Sym_m(H_1(\Sigma_{g+g',1}))).
\]

% Note also that $H_*\pa{\Gamma_{\infty,1}}\subset H_*\pa{\Gamma_{\infty,1};\Sym_{\bullet}(\H)}$ in a natural way.
The right-hand side $H_*\pa{\Gamma_{\infty,1}}\otimes \pa{\bigoplus_{m\geq 0} H_*\pa{\mathfrak{S}_m;\Z_2[x_1,\dots,x_m]}}$
in Equation \ref{eq:doubleiso} is a free module over the ring $H_*\pa{\Gamma_{\infty,1}}\otimes H_*(C_{\bullet}(\D))$.
As a consequence we have that $H_*\pa{\Gamma_{\infty,1};\Sym_{\bullet}(\H_{\infty})}$ is a free module
over $H_*\pa{\Gamma_{\infty,1}}$; this last statement is of independent interest, and regards only the homology
of the group $\Gamma_{\infty,1}$ with twisted coefficients in $\Sym_{\bullet}(\H_{\infty})$.

% Both modules are free over the ring $H_*\pa{\Gamma_{\infty,1}}$, so we obtain an isomorphism of $H_*(C_{\bullet}(\D))$-modules
% \[
%  H_*(C_{\bullet}(\D))\otimes H_*\pa{\Gamma_{\infty,1};\Sym_{>0}(\H)}\cong\bigoplus_{m\geq 0} H_*\pa{\mathfrak{S}_m;\Z_2[x_1,\dots,x_m]}.
% \]
% Note also that the right-hand side of Equation \ref{eq:doubleiso} does not depend on the stable mapping class group $\Gamma_{\infty,1}$.
% Here $\Sym_{>0}(\H)=\bigoplus_{m\geq 1}\Sym_m(\H)$.

We turn now to Theorem \ref{thm:BoTtwo}, and for the moment we assume again that $\F$ is a generic field and that the genus $g$
is fixed and finite. Considering
all values of $m$ at the same time, we can equivalently write
\begin{equation}
\label{eq:epsilonsplitting}
 H_*\pa{C_{\bullet}(\cF_{\S,\D});\F}\simeq \F[\epsilon]\otimes_{\F}\pa{\bigoplus_{m\geq 0} H_*\pa{C_m(\cF_{\S,\D}),C_{m-1}(\cF_{\S,\D})};\F}.
 \end{equation}

Here we use the convention $C_{-1}(\cF_{\S,\D})=\emptyset$, and we regard $C_{m-1}(\cF_{\S,\D})$ as a subspace of $C_m(\cF_{\S,\D})$ by using the map
$\mu^{\cF}$ in the statement of Theorem \ref{thm:BoTtwo}, with first input any fixed point $*\in C_1(\D)$. The class
$\epsilon$ is the canonical generator of $H_0(C_1(\D);\F)$.

In the case $\F=\Z_2$, Theorem \ref{thm:mainreformulation} improves Equation \ref{eq:epsilonsplitting} by exhibiting
$ H_*\pa{C_{\bullet}(\cF_{\S,\D})}$ as a free module over the ring $H_*(C_{\bullet}(\D))\cong\Z_2[Q^j\epsilon|j\geq 0]$,
rather than its subring $\Z_2[\epsilon]$.