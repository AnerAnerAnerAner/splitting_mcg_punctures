\section{Introduction}
Let $\sg$ be a smooth orientable surface of genus $g$ with one boundary curve $\partial\sg$, and let $\sgm$ be $\sg$ with
a choice of $m$ distinct points in the interior, called \emph{punctures}.

Let $\gg$ be the mapping class group of $\sg$, i.e. the group of isotopy classes of diffeomorphisms of $\sg$:
diffeomorphisms are required to fix $\partial\sg$ pointwise. Similarly let $\ggm$ be the mapping class group of $\sgm$, i.e.
the group of isotopy classes of diffeomorphisms of $\sgm$ that fix $\partial\sgm$ pointwise and \emph{permute} the $m$ punctures.

Forgetting the punctures gives a surjective map $\ggm\to\gg$ with kernel $\bms$,
the $m$-th \emph{braid group} of the surface $\sg$. We obtain the Birman exact sequence (see \cite{Birman:mcgbr})
\begin{equation}
\label{eq:Birman}
1\to\bms\to\ggm\to\gg\to 1.
\end{equation}

The associated Leray-Serre spectral sequence $E(m)$ in $\Z_2$-homology has a second page $E(m)^2_{k,q}=H_k(\gg;H_q(\bms;\Z_2))$,
and converges to $H_{k+q}(\ggm;\Z_2)$.

The main result of this article is that this spectral sequence collapses in its second page.
\begin{thm}
\label{thm:main}
For all $l\geq 0$ there is an isomorphism of vector spaces
\begin{equation}
\label{eq:main}
H_l\pa{\ggm;\Z_2}\cong \bigoplus_{k+q=l} H_k\pa{\gg;H_q\pa{\bms;\Z_2}}.
\end{equation}
\end{thm}
Thus the computation of $H_*\pa{\ggm;\Z_2}$ reduces to the computation of the homology of $\gg$ with
twisted coefficients in the representation $H_*\pa{\bms;\Z_2}$. We will see that this $\gg$-representation
splits as a direct sum of symmetric powers of $H_1(\sg)$ with the symplectic action: this is done in
Theorem \ref{thm:Hbms*as*ggrep}, which together with Theorem \ref{thm:main} is the main result of the article.

The strategy of the proof does not generalize to fields of characteristic different from 2 or to the \emph{pure}
mapping class group, in which we consider only diffeomorphisms of $\sgm$ that fix all punctures. In Section \ref{sec:rational}
we describe in detail a counterexample with coefficients in $\Q$, which can be generalized both to coefficients
in a field $\F_p$ of odd characteristic and to the pure mapping class group.

I would like to thank my PhD advisor Carl-Friedrich B\"odigheimer for his precious suggestions and his continuous encouragement
during the preparation of this work.