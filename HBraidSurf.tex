\section{Homology of configuration spaces of surfaces}
In this section we compute the homology $H_*(\bms)=H_*(C_m(\S))$; we
are interested in these groups as they appear in the Leray-Serre spectral sequence associated to
the Birman exact sequence \ref{eq:Birman} or, which is the same, to the bundle
\ref{eq:Birmanbundle}. In particular we are interested in $H_*(\bms)=H_*(C_m(\S))$
\emph{as a $\Z_2-$representation of the group $\gg$.} From now on $\Z_2-$coefficients in homology
and cohomology will always be understood.

In \cite{LM} L\"offler and Milgram proved that this must be a symplectic
representation of the mapping class group. By symplectic we mean the following:
\begin{defn}
 \label{def:symplrep}
 Let $\H=H_1(\S)\simeq\Z_2^{2g}$.
 The natural action of $\gg$ on $\H$ induces a surjective map
 $\gg\to Sp_{2g}(\Z_2)$. A representation of $\gg$ over $\Z_2$ is called \emph{symplectic}
 if it is a pull-back of a representation of $Sp_{2g}(\Z_2)$ along this map.
\end{defn}

In \cite{BCM} B\"odigheimer, Cohen and Taylor computed $H_*(C_m(\S))$ \emph{as a graded $\Z_2-$vector space}.
Their method provides all Betti numbers, but the action of $\gg$ cannot be easily deduced, as
their descripition of $H_*(C_m(\S))$ depends on a handle decomposition of $\S$.

We prove here the following theorem that, to the best of the author's knowledge, doesn't appear
in the literature.
\begin{thm}
 \label{thm:Hbms*as*ggrep}
 There is an isomorphism of bigraded $\Z_2-$representations of $\gg$
 \[
  H_*\pa{\coprod_{m\geq 0} C_m(\S)}\simeq \Z_2\left[Q^i\epsilon\,|\, i\geq 0\right]\otimes\Sym_{\bullet}(\H).
 \]
 Here we mean the following:
 \begin{itemize}
  \item the bigrading is given on left by homological degree $*$ and by the connected component
  in the disjoint union on which the homology class is supported, i.e. by the number $m$ of points
  involved in constructing the homology class; we call $*$ the \emph{degree} and $m$ the \emph{weight};
  \item for $i\geq 0$ $Q^i\epsilon$ is the image in $H_{2^i-1}(C_{2^i}(\S))$ of a generator
  of the group $H_{2^i-1}(C_{2^i}(\Sigma_{0,1}))\simeq \Z_2$ 
  under the natural map induced by an embedding $\Sigma_{0,1}\hookrightarrow\S$,
  and $\Z_2\left[Q^i\epsilon\,|\, i\geq 0\right]$ is the polynomial ring on
  infinitely many variables $\epsilon,Q\epsilon,Q^2\epsilon,\dots$;
  \item $\H$ is identified with $H_1(C_1(\S))$ in a natural way, and $\Sym_{\bullet}(\H)$ is the
  symmetric algebra on $\H$;
  \item degrees and weights are extended on right by the usual multiplicativity-additivity rule;
  \item the action of $\gg$ on right is the diagonal action on the product: it is a trivial action
  on the factor $\Z_2[Q^i\epsilon\,|\, i\geq 0]$ and it is the symplectic action induced by the action on $\H$ on the
  factor $\Sym_{\bullet}(\H)$.
  \end{itemize}
  
  Notice that for any bi-homogeneus element in the right-hand side, the weight is greater or equal than
  the degree.
\end{thm}
For the rest of the section we will write $\S=\S$.

\subsection{Cohomology of $C_m(\gg)$.} Our first aim is to compute the \emph{cohomology} $H^*(C_m(\gg))$ as a graded
$\Z_2-$vector space (not as a $\gg-$representation yet); we will mimic the approach of Fuchs (\cite{Fuchs:CohomBraidModtwo}).
We will turn back to homology of $\cms$ in the next subsection.

As already said this computation recovers a known result, but it has the advantage of
being quite elementary and of providing a part of
the geometric insight that we will need later.

We first introduce a space $\T(\S)$ which is homeomorphic to the interior of $\S$. The construction
corresponds to a handle decomposition of $\S$ with one $0-$handle and $2g$ $1-$handles.
\begin{defn}
\label{def:Tsg}
If $g=0$, hence $\S=\sg$ is the disc, we set $\T(\S)=(0,1)^2$, the interior of the unit square. Assume
now $g\geq 1$.

Dissect the interval $[0,1]$ into $2g$ equal subintervals through the points $P_i=\frac{i}{2g}$ for $0\leq i\leq 2g$
(for $i=0,2g$ we get the two endpoints of $[0,1]$).

Let $Q$ be the subspace of $[0,1]^2$ obtained by the union of $(0,1)^2$ and the vertical open intervals
$I_i^l=\set{0}\times (P_i,P_{i+1})$ and $I_i^r=\set{1}\times (P_i,P_{i+1})$ for $1\leq i\leq 2g$;
equivalently, we remove from $[0,1]^2$ the two horizontal sides
$[0,1]\times\set{0,1}$ and the $4g-2$ points $\set{0,1}\times \set{P_1,\dots,P_{2g-1}}$.

Notice that all intervals $I_i^l$'s and $I_j^r$'s are
canonically diffeomorphic
to $(0,1)$ by projecting on the second coordinate, rescaling linearly by $2g$
(without changing orientation!) and translating; therefore we will specify a bijection
between the two sets of left and right intervals, and then $\T(\S)$ will be obtained from $Q$
by identifying in the canonical way the two intervals in each couple.

For $1\leq i\leq g$, we identify $I^l_{2i-1}$ with $I^r_{2i}$, obtaining an open interval $\U_i\subset\T(\S)$,
and we identify $I^r_{2i-1}$ with $I^l_{2i}$, obtaining an open interval $\V_i\subset\T(\S)$.
Each interval $\U_i,\V_i$ has a natural parametrisation by $(0,1)$.
\end{defn}
The space $\T(\S)$ is homeomorphic to $\mrS$; from now
on we will identify the two open surfaces and in particular we will identify $\cms$ with the
space of configurations of $m$ points in $\T(\S)$.

We consider the one-point-compactification $\cms^c$ of $\cms$; since the space $\cms$ is the interior of a compact
$2m-$manifold with boundary, by Poincaré-Lefschetz
duality we have
\[
 H^*(\cms)\simeq \tilde H_{2m-*}(\cms^c).
\]
Our next aim is to define a cell structure on the space $\cms^c$.
\begin{defn}
\label{def:ehopen}
A \emph{tuple} $\tup$ is a choice of the following set of data:
 \begin{itemize}
  \item a natural number $0\leq l\leq m$;
  \item integers $x_1,\dots,x_l\geq 1$;
  \item integers $u_1,\dots,u_g$ and $v_1,\dots,v_g$, all $\geq 0$.
 \end{itemize}
with the condition that
\[
 m=\sum_{i=1}^lx_i+\sum_{i=1}^g(u_i+v_i).
\]
% We will generically use the letter $m$ to denote any of the numbers $l$, $x_i$,$y_i$,$z$,$u_i$,$v_i$ or $w_i$
% in the above sum, so $m\in\Z$; instead .
% Similarly we will use the letter $\M$ to denote any of the intervals $\U_i,\V_i,\W_i$;
% they will correspond to the letters $U_i,V_i,W_i$, so we also define a subset
% $\bsymb=\set{(U_i,V_i)_{i\leq g},(W_i)_{i\leq n-1}}\subset\symb$.

The \emph{degree} of $\tup$ is defined as $m+l$.

For a tuple $\tup$ let $e^{\tup}$ be the subspace
of $\cms$ of configurations of $m$ points in $\mrS$ such that
\begin{itemize}
 \item for all $1\leq i\leq g$, exactly $u_i$ points lie on $\U_i$
 and exactly $v_i$ points lie on $\V_i$;
 \item there are exactly $l$ vertical lines in $(0,1)^2\subset\mrS$ of the
 form $\set{s_i}\times(0,1)$ for some $0<s_1<\dots<s_l<1$, containing at least one
 point of the configuration. From left to right, these lines contain exactly $x_1,\dots,x_l$ points
 respectively.
\end{itemize}
\end{defn}
The space $e^{\tup}$ is homeomorphic to \emph{the interior} of the following product of simplices:

% We want to show that the collection of the $e_h$ for varying $h$, together with
% the 0-cell $*$, give a cell decomposition of $C_k(\S)^c$. Denote by $\Delta^h$ the following product of simplices
\[
 \Delta^{\tup}\colon =
 \Delta^l\times\prod_{i=1}^l\Delta^{x_i}\times\prod_{i=1}^g\pa{\Delta^{u_i}\times\Delta^{v_i}},
%  =\prod_{m\in\symb_l}\Delta^m.
\]
where the simplex $\Delta^r$ is the subspace of $[0,1]^r$ of sequences $0\leq \tau_1\leq\dots\leq\tau_r\leq 1$
(the numbers $\tau_1,\dots,\tau_r$ are the \emph{local coordinates} of the simplex). The homeomorphism is given
as follows:
\begin{itemize}
 \item the local coordinates of the $\Delta^l-$factor correspond to the positions $s_1,\dots,s_l$ of the vertical
 lines in $(0,1)^2$ containing points of the configuration;
 \item the local coordinates of the $\Delta^{x_i}-$factor correspond to the positions of the $x_i$ points
 lying on the vertical line $\set{s_i}\times(0,1)$;
 \item the local coordinates of the $\Delta^{u_i}-$factor correspond to the positions of the $u_i$ points
 lying on $\U_i$, which is canonically identified with $(0,1)$; similarly for the $\Delta^{v_i}-$factor,
 with $v_i$ and $\V_i$ instead of $\U_i$ and $u_i$.
\end{itemize}
Notice that $\Delta^{\tup}$ has dimension equal to the degree of $\tup$.
This embedding $\mathring{\Delta}^{\tup}\cong e^{\tup}\hookrightarrow \cms^c$ extends to a continuous map
$\phi^{\tup}\colon\Delta^{\tup}\to\cms^c$, so that the image of $\partial\Delta^{\tup}$ is contained in the union of
subspaces $e^{\tup'}$ for tuples $\tup'$ with \emph{lower} degree than $\tup$, together with the point
at infinity.

The construction of the map $\phi^{\tup}$ is as follows:
\begin{enumerate}
\item we see the one-point-compactification
$\mrS^c$ of $\mrS=\T(\S)$ as the quotient of $[0,1]^2$ that identifies left and right intervals as in
in definition \ref{def:Tsg} of $\T(\S)$, and moreover collapses $[0,1]^2\setminus Q$ to a point, which
is the point at infinity;
\item we consider the $m-$fold symmetric product $\SP^m(\mrS^c)$: it contains 
as an \emph{open} subspace $\cms$, so we can see $\cms^c$ as the quotient of $\SP^m(\mrS^c)$ by
the subspace $\SP^m(\mrS^c)\setminus\cms$;
\item the homeomorphism $\mathring{\Delta}^{\tup}\to e^{\tup}\subset\cms$ extends now to a map
$\Delta^{\tup}\to \SP^m(\mrS^c)$, that we can then further project to $\cms^c$: the composition
is the map $\phi^{\tup}$.
\end{enumerate}

Therefore the collection of the $e^{\tup}$, together with the $0-$cell given by the point at infinity,
give a cell decomposition of $\cms^c$, with characteristic maps of cells $\phi^{\tup}$.

We can easily use these characteristic maps to compute the \emph{reduced} cellular chain complex of $\cms^c$
with coefficients in $\Z_2$, which
reads as follows.
\begin{lem}
\label{lem:doperatoropenmodtwo}
Let $\tup=(l,(x_i)_{i\leq l},(u_i,v_i)_{i\leq 2g})$ and
$\tup'=(l-1,(x'_i)_{i\leq l-1},(u'_i,v'_i)_{i\leq 2g})$
be tuples in consecutive degrees $m+l$ and $m+l-1$, and let $[\partial \tup\colon\tup']\in\Z_2$ denote the coefficient
of $\tup'$ in $\partial \tup$ in the chain complex $\tilde\Ch_*(\cms^c)$.
Then $[\partial \tup:\tup']=0$ unless one (and exactly one) of the following situations occur:
\begin{itemize}
 \item $l\geq 2$ and $\tup'$ is obtained from $\tup$ by decreasing $l$ by 1, setting $x'_i=x_i+x_{i+1}$
 for one value $1\leq i\leq l-1$, and shifting the values
 $x'_j=x_{j+1}$ for $i+1\leq j\leq l-1$. In this case
 \[
  [\partial \tup\colon\tup']=\binom{x_i+x_{i+1}}{x_i}.
 \]
 This binomial coefficient may be anyway equal to $0\in\Z_2$.
 \item $l\geq 1$ and $\tup'$ is obtained from $\tup$ by decreasing $l$ by 1, choosing a splitting of $x_1$
 in integers $\delta u_i,\delta v_i\geq 0$
 \[
  x_1=\sum_{i=1}^g \pa{\delta u_i+\delta v_i},
 \]
 setting $u'_i=u_i+\delta u_i$ and $v'_i=v_i+\delta v_i$ for all $1\leq i\leq g$ and shifting the values
 $x'_j=x_{j+1}$ for $1\leq j\leq l-1$. In this case
 \[
  [\partial \tup\colon\tup']=\prod_{i=1}^g\pa{\binom{u_i+\delta u_i}{u_i}\binom{v_i+\delta v_i}{v_i}}.
 \]
 This binomial coefficient may be anyway equal to $0\in\Z_2$.
 \item $l\geq 1$ and $\tup h$ is obtained from $\tup$ by decreasing $l$ by 1, choosing a splitting of $x_l$
 in integers $\delta u_i,\delta v_i\geq 0$
 \[
  x_l=\sum_{i=1}^g \pa{\delta u_i+\delta v_i},
 \]
 setting $u'_i=u_i+\delta u_i$ and $v'_i=v_i+\delta v_i$ for all $1\leq i\leq g$ and keeping $x'_i=x_i$ for all $1\leq i\leq l-1$.
 Also in this case
 \[
  [\partial \tup\colon\tup']=\prod_{i=1}^g\pa{\binom{u_i+\delta u_i}{u_i}\binom{v_i+\delta v_i}{v_i}}.
 \]
 This binomial coefficient may be anyway equal to $0\in\Z_2$.
\end{itemize}
\end{lem}
\begin{proof}
 If we restrict the map $\phi^{\tup}\colon\Delta^{\tup}\to\cms$ to any face of the multisimplex
 $\Delta^{\tup}$ coming from a factor different from $\Delta^l$,
 we obtain either a constant map to the point at infinity, hence these faces don't contribute
 to the boundary of $e^{\tup}$ in the reduced cellular chain complex. Instead for $0\leq i\leq l$ the restriction
 \[
  \phi^{\tup}\colon\partial_i\Delta^l\times\prod_{i=1}^l\Delta^{x_i}\times\prod_{i=1}^g\pa{\Delta^{u_i}\times\Delta^{v_i}}\to\cms^c
 \]
 hits the cell $e^{\tup'}$ exactly as many times as specified in the statement of the lemma for the cases $1\leq i\leq l-1$,
 $i=0$ and $i=l$ respectively. We are working in $\Z_2$ so we don't have to be careful about orientations of simplices
 and therefore about signs.
\end{proof}

We can filter the chain complex $\tilde\Ch_*(\cms^c)$ by giving weight $\sum_{i=1}^lx_i$ to
the cell $e^{\tup}$, with $\tup=(l,(x_i)_{i\leq l},(u_i,v_i)_{i\leq g})$. By lemma \ref{lem:doperatoropenmodtwo}
the weight is weakly decreasing along boundaries. Let $F_p\subset\tCh_*(\cms^c)$ be the subcomplex of cells
of weight $\leq p$, and let $F_p/F_{p-1}$ be the $p-$th filtration stratum.

Then $F_p/F_{p-1}$ is isomorphic, as a chain complex, to a direct sum of copies of $\tCh_*(C_p(\Sigma_{0,1}))$,
one copy for each way of splitting $m-p=\sum_{i=1}^g (u_i+v_i)$ with $u_i,v_i\geq 0$, where the isomorphism
shifts degrees by $p$. Indeed in $F_p/F_{p-1}$ we can compute differentials only according to the first
of the three possibilities listed in lemma \ref{lem:doperatoropenmodtwo}, in particular the numbers $u_i,v_i$
don't change along boundaries in $F_p/F_{p-1}$; it is then immediate to identify the \emph{internal} differentials
with the ones one would get in the case $g=0$.

We notice that $\tCh_*(C_p(\Sigma_{0,1}))$ is exactly the chain complex described by Fuchs in
\cite{Fuchs:CohomBraidModtwo}; in particular its homology is generated by symmetric chains,
where, for a splitting $p=\sum_{j=0}^{\infty}\alpha_j2^j$ of $p$ into powers of 2, with multiplicities $\alpha_i$,
the associated symmetric chain is the sum of all $e^{\tup}$ where $\tup$ ranges among all tuples
$(l,(x_i)_{i\leq l})$ such that
\begin{itemize}
 \item $l=\sum_{i=1}^{\infty}\alpha_i$;
 \item every $x_i$ is a power of 2;
 \item for all $j\geq 0$ there are exactly $\alpha_j$ indices $i$ such that $x_i=2^j$.
\end{itemize}

If we run the Leray spectral sequence associated to the filtered chain complex $\tCh_*(\cms)$,
the $E^1-$page contains many copies of the homology of $\tCh_*(C_p(\Sigma_{0,1}))$,
each generated by a symmetric chain. It is straightforward to check that these symmetric
(arising as cycles in $F_p/F_{p-1}$), are also cycles when lifted, as a sum of the same
cells, to the complex $\tCh_*(\cms^c)$: one has only to check that the \emph{outer}
differentials cancel out, and this is easy to see as every \emph{left} differential
of one cell in the sum of a symmetric chain corresponds to a \emph{right} differential
of possibly another cell in the same symmetric chain.

In particular the spectral sequence collapses on its first page and the $\Z_2$-homology of
$\cms^c$ has a $\Z_2$-basis given by all choices of the following set of data:
\begin{itemize}
 \item a number $0\leq p\leq m$;
 \item a splitting $m-p=\sum_{i=1}^g (u_i+v_i)$ with $u_i,v_i\geq 0$;
 \item a splitting $p=\sum_{j=0}^{\infty}\alpha_j2^j$ with $\alpha_j\geq 0$.
\end{itemize}
The class $(p,(u_i,v_i),(\alpha_j))\in H_*(\cms^c)$ has homological degree
$m+\sum_{j=1}^{\infty}\alpha_i$.

% To interpret this class
% as a monomial in the tensor product of theorem \ref{thm:Hbms*as*ggrep} we need the following definition.
\begin{defn}
 \label{defn:dualHbasis}
We can see the $\U_i$'s and $\V_i$'s as properly embedded $1-$manifolds in $\mrS=\T(\S)$;
by Poincaré-Lefschetz duality they represent classes in $H^1(\S)$, and in particular they form a basis
of this cohomology group. We fix simple closed curves $\u_i,\v_i$ on $\S$, for $1\leq i\leq g$, whose fundamental classes
form the dual basis of $H_1(\S)$. We assume that, apart from the following necessary exceptions,
all the curves $\U_i,\V_i,\u_i,\v_i$ for $1\leq i\leq g$ are disjoint:
\begin{itemize}
 \item $\u_i$ and $\v_i$ intersect once, transversely;
 \item $\u_i$ and $\U_i$ intersect once, transversely;
 \item $\v_i$ and $\V_i$ intersect once, transversely;
 \end{itemize}
SEE PICTURE.
\end{defn}
We establish a bijection between monomials in the tensor product of theorem  \ref{thm:Hbms*as*ggrep}
and the basis of $H^*(\cms)\simeq H_*(\cms^c)$ that we have just found.

Then the class 
\[
(p,(u_i,v_i),(\alpha_j))\in H_{m+\sum\alpha_j}(\cms^c)\simeq H^{m-\sum\alpha_j}(\cms)
\]
is associated with the monomial $\prod_{j=1}^{\infty}(Q^j\epsilon)^{\alpha_j}\otimes \prod_{i=1}^g([\u_i]^{u_i}[\v_i]^{v_i})$.

This shows an isomorphism of bigraded vector spaces
\begin{equation}\label{eq:isovectorspaces}
  \bigoplus_{m\geq 0} H^*(\cms)\simeq \Z_2\left[Q^i\epsilon\,|\, i\geq 0\right]\otimes\Sym_{\bullet}(\H),
\end{equation}

and using the canonical identification of $\Z_2-$cohomology as the dual of homology, we can
conclude that there exists an isomorphism as in theorem \ref{thm:Hbms*as*ggrep}
at least \emph{as bigraded $\Z_2-$vector spaces}.

\subsection{Action of $\gg$} We now turn back to \emph{homology} of $\cms$;
our first task is to construct a map
\[
\psi_m\colon \Sym_m(\H))\to H_m(\cms).
\]
% the former group is isomorphic to $\Sym_m(H^1(\S))$ and the second to $H^m(\cms)$,
% both by Poincaré-Lefschetz duality.

We already know that these $\Z_2-$vector spaces have the same dimension:
for the dimension of $H_m(\cms)$ we have to count the monomials of degree $m$ and weight $m$
in the right-hand side of equation \ref{eq:isovectorspaces}; a monomial whose
weight is \emph{equal} and not bigger than its degree cannot contain factors of the form
$Q^i\epsilon$, i.e. it should lie in the factor $\Sym_{\bullet}(\H)$, and if the degree
is $m$ it should lie in $\Sym_m(\H)$.

The map $\psi_m$ that we construct will turn out to be a $\gg-$equivariant isomorphism.
We will first construct a map $\tpsi_m\colon \H^{\otimes m}\to H_m(\cms)$ and then show
that factors through $\Sym_m(\H)$.

% \begin{defn}
% \label{defn:cCm}
% Let $\cC^m$ be the the (discrete) set of isotopy classes of $m-$tuples of closed curves $c_1,\dots,c_m\subset\mrS$
% such that:
% \begin{itemize}
%  \item each $c_i\colon\Sone\to\mrS$ is smooth and a local embedding;
%  \item the curves $\set{c_i}_{i\leq m}$
%  are pairwise transverse, no three of them intersect at a common point in $\mrS$.
% \end{itemize}
% Two such $m-$tuples are isotopic if there is an ambient isotopy of $\S$
% transforming one $m-$tuple into the other. For each isotopy class we
% suppose to have fixed a representative $(c_1,\dots,c_m)$ made of curves that are
% also transverse to all $\U_i$'s and $\V_i$'s.
% 
% We denote by $\ZcC{m}$ the free $\Z_2-$vector space with basis $\cC^m$.
% There is a canonical surjective map $\pi\colon\ZcC{m}\to\Sym_m(\H)$ given by
% \[
%  \pi(c_1,\dots,c_m)=[c_1]\cdot\ldots\cdot[c_m].
% \]
% The group $\gg$ acts both on $\ZcC{m}$, by acting on its basis $\cC^m$, and
% on $\Sym_m(\H)$, symplectically. The map $\pi$ is $\gg-$equivariant.
% \end{defn}
% Our aim is to define a $\gg-$equivariant map $\tpsi\colon\ZcC{m}\to H_m(\cms)$ and show that it
% factors through $\pi$ to give the desired isomorphism $\psi$.
% 
% The rough idea is to map the basis element
% $(c_1,\dots,c_m)$ first to $[c_1]\otimes\dots\otimes[c_m]\in H_m((\S)^{\times m})$, then map
% this further to $[c_1]\cdot\ldots\cdot[c_m]\in H_m(\SP^m(\S))$ and then lift this to an element in
% $H_m(\cms)$ along the inclusion $\cms\subset \SP^m(\S)$.
% 
We construct $\tpsi_m$ by induction on $m$. For $m=0$ both source and target are
isomorphic to $\Z_2$; for $m=1$
both source and target are canonically identified with $\H$: we take $\tpsi_0$ and $\tpsi_1$
to be these canonical isomorphisms.
% we simply map $c_1$ to its fundamental class in $\H=H_1(\S)=H_1(C_1(\S))$.
Let now $m\geq 1$ and suppose that we have defined a map $\tpsi_m\colon \H^{\otimes m}\to H_m(\cms)$.

\begin{defn}\label{defn:variationsCm}
Let $\D\subset\mrS$ be the open square $(1/4;3/4)\times(1/2,1)$, using the model $\T(\S)$:
$\D$ is an open disc in $\mrS$ near the boundary and disjoint from all $\U_i,\V_i$.

We introduce a few variations of the notion of configuration spaces.
\begin{itemize} 
 \item The space $C_{1,m}(\S)$ is the subspace of $\mrS\times \cms$ of elements
 $(q;\set{p_1,\dots,p_m})$ with $q\neq p_i$ for all $i$; in other words it is
 the space of configurations of $m+1$ points, one of which is \emph{red} (meaning that
 it is special), whereas the other are \emph{black}.
 \item The space $\cmstwo$ is the subspace of $\SP^m(\mrS)$ of configurations where either all $m+1$ points
 are distinct, or there is exactly one point with multiplicity $2$ and $m-1$ other points with multiplicity $1$:
 in the latter case we require the double point to lie in $\D$.
\end{itemize}
There is a natural map $\pi\colon\mrS\times \cms\to \cmstwo$ , which converts the red point
into a black point. This map restricts to a covering map $C_{1,m}(\S)\to C_{m+1}(\S)$; we have a commutative
 diagram
 \begin{equation}\label{eq:cmstwodiagram}
  \begin{CD}
   C_{1,m}(\S) @>\subset >> \mrS\times \cms \\
   @V\pi VV @V\pi VV\\
   C_{m+1}(\S) @>\subset >> \cmstwo
   \end{CD}
\end{equation}
We notice that all these spaces are $2(m+1)-$dimensional manifolds, arising as open subsets
of $\mrS\times\SP^m(\mrS)$ or $\SP^{m+1}(\mrS)$, which are manifolds of dimension $2(m+1)$.

Let $\Delta=\cmstwo\setminus C_{m+1}(\S)$ and similarly $\tilde\Delta=\mrS\times \cms\setminus C_{1,m}(\S)$.
We notice that $\pi$ restricts to a homeomorphism $\tilde\Delta\to\Delta$. Moreover both $\Delta\subset\cmstwo$
and $\tilde\Delta\subset\mrS\times \cms$ are embedded submanifolds of codimension $2$, and the map $\pi$ restricts to
a $2-$ramified covering between the normal bundles, which are bundles with fiber a $2-$disc.
\end{defn}

Diagram \ref{eq:cmstwodiagram} induces a commutative diagram in homology
\[
\minCDarrowwidth15pt
 \begin{CD}
  @. H_{m+1}\pa{\mrS\times \cms} @>>> H_{m+1}\pa{\mrS\times \cms, C_{1,m}(\S)}\\
  @. @V\pi_*VV @V\pi_*VV\\
  H_{m+1}\pa{ C_{m+1}(\S)} @>>> H_{m+1}\pa{\cmstwo} @>>> H_{m+1}\pa{\cmstwo,C_{m+1}(\S)}
 \end{CD}
\]

We can map $\H^{\otimes m+1}$ to $H_{m+1}\pa{\mrS\times \cms}$ through the map
$\Id_{\H}\otimes\tpsi_m$, and then further down to $H_{m+1}\pa{\cmstwo}$.

We claim that
we land in the image of the map
\[
H_{m+1}\pa{ C_{m+1}(\S)}\to H_{m+1}\pa{\cmstwo}.
\]
To check this we further
map towards right to $H_{m+1}\pa{\cmstwo,C_{m+1}(\S)}$ and prove that we get the zero map.
Indeed the other equivalent composition
involves the map
\[
\pi_*\colon H_{m+1}\pa{\mrS\times \cms, C_{1,m}(\S)}\to H_{m+1}\pa{\cmstwo,C_{m+1}(\S)};
\]
this map can be rewritten, after using the Thom isomorphism, as a map
\[
 H_{m-1}(\tilde\Delta)\to H_{m-1}(\Delta)
\]
and the latter map is multiplication by $2$, as the normal bundle of $\tilde\Delta$ is a
double covering of the normal bundle of $\Delta$. Therefore the claim is proved, and
as we are working with $\Z_2-$vector spaces there exists a lift
\[
 \tpsi_{m+1}\colon\H^{\otimes m+1}\to H_{m+1}(C_{m+1}(\S))
\]
of the map $\pi_*\circ(\Id_{\H}\otimes\tpsi_m)\colon \H^{\otimes m+1}\to H_{m+1}\pa{\cmstwo}$.

Thus we have constructed, for each $m\geq 0$, a map $\tpsi_m\colon \H^{\otimes m}$.



\begin{lem}
 \label{lem:cmstocmstwoinjective}
 The map
 \[
H_m\pa{\cms}\to H_m\pa{C_m^{(2)}}
\]
is injective
 \end{lem}
 \begin{proof}
  We can equivalently check that the map $H^m\pa{\cmstwo}\to H^m\pa{\cms}$ is surjective,
  or, again, that the map $H_m(\cmstwo^c)\to H_m(\cms^c)$ is surjective.
  
  The last statement is clear
 \end{proof}

