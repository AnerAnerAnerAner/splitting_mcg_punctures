\documentclass{amsart}
\usepackage{amsmath,amssymb,amsthm,amsfonts,amscd,mathrsfs,mathtools}
\usepackage[utf8]{inputenc}
\usepackage{graphicx}
\usepackage{enumitem}
\usepackage[hyphens]{url}
\usepackage{caption}
\usepackage{hyperref}
\usepackage{tikz-cd}

\setlength\parindent{0pt}

\hyphenation{e-qui-va-riant se-cond ho-mo-lo-gy co-ho-mo-lo-gy}
% \setlength{\parindent}{6pt}

\theoremstyle{plain}
    \newtheorem{thm}{Theorem}[section]
    \newtheorem{lem}[thm]   {Lemma}
    \newtheorem{cor}[thm]   {Corollary}

\theoremstyle{definition}
    \newtheorem{defn}[thm]  {Definition}
    \newtheorem{conj}[thm]{Conjecture}
    \newtheorem{ex}{Example}

\newcommand{\C}{\mathbb{C}}
\newcommand{\cC}{\mathfrak{C}}
\newcommand{\fc}{\mathfrak{c}}
\newcommand{\D}{\mathcal{D}}
\newcommand{\F}{\mathbb{F}}
\newcommand{\cF}{\mathcal{F}}
\renewcommand{\H}{\mathcal{H}}
\newcommand{\M}{\mathcal{M}}
\newcommand{\N}{\mathcal{N}}
\newcommand{\p}{p}
\newcommand{\cP}{\mathcal{P}}
\newcommand{\Q}{\mathbb{Q}}
\newcommand{\R}{\mathbb{R}}
\renewcommand{\S}{\mathcal{S}}
\newcommand{\fS}{\mathfrak{S}}
\newcommand{\T}{\mathcal{T}}
\newcommand{\U}{\mathcal{U}}
\renewcommand{\u}{\mathfrak{u}}
\newcommand{\V}{\mathcal{V}}
\renewcommand{\v}{\mathfrak{v}}

\newcommand{\Z}{\mathbb{Z}}


\newcommand{\sgm}{\Sigma_{g,1}^m}
\newcommand{\sg}{\Sigma_{g,1}}

\newcommand{\Sg}{\mathcal{S}_{g,1}}
\newcommand{\Sgm}{\Sg^m}

\newcommand{\mg}{\mathfrak{M}_{g,1}}
\newcommand{\mgm}{\mg^m}
\newcommand{\Ig}{\mathcal{I}_{g,1}}

\renewcommand{\gg}{\Gamma_{g,1}}
\newcommand{\ggm}{\gg^m}

\newcommand{\bms}{\beta_{m}(\sg)}
\newcommand{\fms}{F_m(\sg)}
\newcommand{\cms}{C_m(\S)}
\newcommand{\fmsD}{C_{1,m}^{\D}(\S)}
\newcommand{\cmsD}{C_{m}^{\D}(\S)}
\newcommand{\fmstwoD}{C_{1,m}^{(2),\D}(\S)}
\newcommand{\cmstwoD}{C_{m+1}^{(2),\D}(\S)}
\newcommand{\Dmone}{\triangle_{m+1}}
\newcommand{\Donem}{\triangle_{1,m}}
\renewcommand{\i}{\iota}
\renewcommand{\j}{j}
\newcommand{\pr}{\mathfrak{p}}

\newcommand{\symb}{\mathcal{S}}
\newcommand{\bsymb}{\bar\symb}
\newcommand{\tup}{\mathfrak{h}}

\newcommand{\bH}{\mathbb{H}}
\newcommand{\cH}{\mathcal{H}}

\newcommand{\Ch}{\mathcal{C}}
\newcommand{\tCh}{\tilde\Ch}
\newcommand{\tH}{\tilde{H}}

\newcommand{\ZcC}[1]{\Z_2\left<\cC^{#1}\right>}
\newcommand{\tpsi}{\tilde\psi}

\newcommand{\pa}[1]{\left(#1\right)}
\newcommand{\qua}[1]{\left[#1\right]}
\newcommand{\abs}[1]{\left|#1\right|}
\newcommand{\set}[1]{\left\{#1\right\}}

\newcommand{\Sone}{\mathbb{S}^1}
\newcommand{\mrS}{\mathring{\S}}
\newcommand{\gone}{\Gamma_{1,1}}
\newcommand{\tu}{\tau_{\u}}
\newcommand{\tv}{\tau_{\v}}

\newcommand{\ux}{\underline{x}}
\newcommand{\uu}{\underline{u}}
\newcommand{\uv}{\underline{v}}
\newcommand{\ualpha}{\underline{\alpha}}

\newcommand{\tildeu}{\tilde u}
\newcommand{\tildev}{\tilde v}

\newcommand{\SP}{S\!P}

\renewcommand{\phi}{\varphi}
\renewcommand{\epsilon}{\varepsilon}

\newcommand{\hor}{hor}
\newcommand{\ver}{ver}

\renewcommand{\L}{\Lambda}
\DeclareMathOperator{\Diff}{Diff}
\DeclareMathOperator{\Imm}{Imm}
\DeclareMathOperator{\rk}{rk}
\DeclareMathOperator{\Ker}{ker}
\DeclareMathOperator{\coker}{coker}
\DeclareMathOperator{\Sym}{Sym}
\DeclareMathOperator{\Id}{Id}
\DeclareMathOperator{\Hom}{Hom}


\DeclarePairedDelimiter\ceil{\lceil}{\rceil}
\DeclarePairedDelimiter\floor{\lfloor}{\rfloor}
\def\colim{\mathop{\mathrm{colim}}\nolimits}
    
\begin{document}

\title{Rational homology of configuration spaces of surfaces}

\author{Andrea Bianchi}

\address{Mathematics Institute,
University of Bonn,
Endenicher Allee 60, Bonn,
Germany
}


\email{bianchi@math.uni-bonn.de}

\date{\today}

\maketitle
Let $C_n(\Sg)$ denote the unordered configuration spaces of $n$ points
on the surface $\Sg$. In this note I list all the properties I know about $H_*\pa{C_n(\Sg);\Q}$.
Many of these properties can be found somewhere in the literature,
From now on rational coefficients in homology are always understood.

Let $C_{\bullet}(\Sg):=\coprod_{n\geq 0}C_n(\Sg)$, which is a module over the 
(homotopy) associative H-space $C_{\bullet}(\C)=\coprod_{n\geq 0}C_n(\C)$.

Then $\M:=H_*(C_{\bullet}(\Sg)$ is a bigraded $\Gamma_{g,1}$-representation over the bigraded Pontryiagin
ring $H_*(C_{\bullet}(\C))\cong\Q[\epsilon]\otimes \Lambda (\alpha)$, where $\epsilon$ has
bidegree (0,1) and $\alpha$ has bidegree (1,2). The bidegree $(*,\bullet)$ is given by
the homological degree $*$ and the weight (number of points used) $\bullet$.

Let $\bH=H_1(\Sg)$, concentrated in bidegree (2,2), and let $\cH=H_1(\Sg)$, concentrated in bidegree (1,1).
The following hold:

\begin{enumerate}
 \item $\M$ is in general neither a free $\Q[\epsilon]\otimes \Lambda (\alpha)$-module (only for $g=0$),
 nor a symplectic $\Gamma_{g,1}$-representation (only for $g\leq 1$, see the \emph{rational counterexample}
 in my article).
 \item Nevertheless $\M$ is a free $\Q[\epsilon]$-module (see work of B\"odigheimer and Tillmann), and the Johnson's kernel
 $\mathcal{J}_{g,1}\subset\Gamma_{g,1}$ acts trivially on it (one can check that
 separating Dehn twists act trivially on a generatring set of $\M$).
 \item Let $\Sym(\bH)$ denote the symmetric algebra on $\bH$, which is a bigraded ring with a
 symplectic action of $\Gamma_{g,1}$. Then $\M$ has a natural structure of (equivariant)
 $\Sym(\bH)$-module, i.e. there is a $\Gamma_{g,1}$-equivariant map
 \[
  \mu\colon\Sym(\bH)\otimes \M\to \M
 \]
 which satisfies the usual associativity, it preserves the bidigree and it is compatible with
 the action of the ring $\Q[\epsilon]\otimes \Lambda (\alpha)$ (this seems true to me, although
 I have never written down a formal proof of it).
 \item One can filter $\M$ by the submodules
 \[
  \M\supset\bH\cdot\M\supset\bH^2\cdot\M\supset\bH^3\cdot \M.
 \]
 Every $\bH^k\cdot\M$ is bigraded, $\Gamma_{g,1}$-invariant, and an equivariant module
 over $\Q[\epsilon]\otimes \Lambda (\alpha)\otimes \Sym(\bH)$. The associated graded,
 denoted $grad\M$, is also a bigraded $\Gamma_{g,1}$-equivariant module over $\Q[\epsilon]\otimes \Lambda (\alpha)\otimes \Sym(\bH)$.
 \item As such, $grad\M$ is isomorphic to the following
 \[
  grad\M\cong \Q[\epsilon]\otimes \Sym(\bH)\otimes [\alpha\otimes\coker(\cdot\wedge\omega)\oplus\Ker(\cdot\wedge\omega)].
 \]
 Here $(\cdot\wedge\omega)\colon\Lambda(\cH)\to\Lambda(\cH)$ is a map of bidegree (-2,-2), given by contracting
 an element of the exterior algebra $\Lambda(\cH)$ with the intersection form of the surface $\omega\in\Lambda_2(\cH^*)$.
 The action of $\Gamma_{g,1}$ is symplectic on $grad\M$. Multiplication by $\alpha$ is trivial on the summand
 $\alpha\otimes\coker(\cdot\wedge\omega)$ and is induced by the natural map
 $\Ker(\cdot\wedge\omega)\subset \Lambda(\cH)\to\coker(\cdot\wedge\omega)$ on the other summand (this should
 also be entailed in the work of Pagaria).
\end{enumerate}
\end{document}




