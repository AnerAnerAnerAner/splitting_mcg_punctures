\documentclass{amsart}
\usepackage{amsmath,amssymb,amsthm,amsfonts,amscd,mathrsfs,mathtools}
\usepackage[hyphens]{url}
\usepackage{hyperref}
\usepackage[a4article]{geometry}
    
\newcommand{\N}{\mathbb{N}}    
\newcommand{\C}{\mathbb{C}}
% \newcommand{\cC}{\mathcal{C}}
% \newcommand{\D}{\mathbb{D}}
% \newcommand{\F}{\mathbb{F}}
% \newcommand{\cF}{\mathcal{F}}
% \newcommand{\cK}{\mathcal{K}}
% \newcommand{\ff}{\mathfrak{f}}
\newcommand{\fM}{\mathfrak{M}}
% \newcommand{\Q}{\mathbb{Q}}
% \newcommand{\R}{\mathbb{R}}
% \renewcommand{\H}{\mathcal{H}}
% \newcommand{\fH}{\mathfrak{H}}
% \newcommand{\V}{\mathcal{V}}
% \newcommand{\Z}{\mathbb{Z}}
% \newcommand{\cS}{\mathcal{S}}
% \newcommand{\bS}{\mathbb{S}}
\newcommand{\fS}{\mathfrak{S}}
% \newcommand{\T}{\mathcal{T}}

% \newcommand{\pa}[1]{\left(#1\right)}
% \newcommand{\abs}[1]{\left|#1\right|}
% \newcommand{\set}[1]{\left\{#1\right\}}


\renewcommand{\phi}{\varphi}
    
\hyphenation{sta-bi-li-ty}


\begin{document}
\title{
Answer to the referee report for ``Splitting of the homology of the
punctured mapping class group''
}
\author{Andrea Bianchi}
\address{Mathematics Institute,
University of Bonn,
Endenicher Allee 60, Bonn,
Germany
}
\email{bianchi@math.uni-bonn.de}
\date{\today}
\maketitle

First, I would like to thank the referee for the meticulous work he/she made
in reviewing my article.

I have accepted almost all corrections and suggested improvements, but in the following cases
I have followed a different way than the one proposed by the referee; I will describe briefly
the reasons for my choice, and I remain open for further discussion.
\begin{itemize}
 \item The referee writes \emph{p.3, line 2: The condition that $g\geq 2$ is needed for $\mathrm{Diff}(\Sigma_{g,1})$ to be contractible.}\\
 Since we are working with surfaces with non-empty boundary, this result should also be true in genera $0$ and $1$; differently
 the components of $\mathrm{Diff}^+(\Sigma_g)$ are not contractible for $g=0,1$. I have briefly reminded the reader of this fact in the article.
 \item The referee writes \emph{p.19, line −11: ``$\mathcal{N}' = (\mathcal{N} \setminus c_1 ) \cup \mathcal{D}$''
 Since $\mathcal{N}$ is a neighborhood of $\mathcal{D} \cup c_1 \cup\dots\cup c_m$, it contains the neighborhood of
 $c_1$. It seems that the author wants to take the complement of the neighborhood
 of $c_1$ in the definition of $\mathcal{N}'$.}\\
 This is not necessary: $\mathcal{N}' = (\mathcal{N} \setminus c_1 ) \cup \mathcal{D}$ is already an open set, contains
 $\mathcal{D} \cup c_2 \cup\dots\cup c_m$ and is disjoint from $c_1\setminus\mathcal{D}$. One can reduce further
 $\mathcal{N}'$ by deleting an entire neighborhood of $c_1$, and then taking again the union with $\mathcal{D}$, but
 this is not necessary for the argument to work. I have stressed in the article that $\mathcal{N}' = (\mathcal{N} \setminus c_1 ) \cup \mathcal{D}$
 is already open.
 \item The referee writes \emph{Section 6: How about mentioning the articles ``Stable cohomology of the mapping
 class group with symplectic coefficients and the universal Abel-Jacobi map'' written by Eduard Looijenga
 and ``Characteristic classes of surface bundles'' written
 by Shigeyuki Morita. They compute the integral stable cohomology ring of the
 mapping class group of punctured surface with boundary.}\\
 It seems to me that Morita's article (BULLETIN   (New  Series)  OF  THE  AMERICAN  MATHEMATICAL   SOCIETY
 Volume  11,  Number  2, pages 386-388,  October   1984) only deals with characteristic classes of
 $\Sigma_g$-bundles (i.e., no puncture, no boundary) over $\mathbb{Q}$. Looijenga's article
 (Journal of Algebraic Geometry. 5. January 1994) also deals mainly with coefficients in $\mathbb{Q}$.
 It seems to me that the only with coefficients in $\mathbb{Z}$ in Looijenga's article is Proposition 2.1:
 however, only the case of \emph{ordered} punctures is treated, whereas the case of unordered
 punctures follows easily only over $\mathbb{Q}$, but, in my opinion, not over $\mathbb{Z}_2$.
 The stable ($g=\infty$), unordered ($m<\infty$ unordered punctures) case with coefficients
 in $\mathbb{Z}_2$ (actually any field) is rather treated in the article of B\"odigheimer and Tillmann,
 which I already cite. Therefore I would leave these two suggested references out. 
\end{itemize}
\end{document}
